% In your .tex file
% !TEX program = xelatex
\documentclass[hyperref={pdfpagelabels=false}, t]{beamer}
\usepackage{ragged2e}
\usepackage{changepage}
\usepackage{lmodern}
\usepackage{graphicx}
%\usepackage{enumitem} % for no bullets
\usetheme{Madrid}
%\providecommand{\thispdfpagelabel}[1]{}
%%-----------------------------------------------------------------------------
% Languages
%%-----------------------------------------------------------------------------
\usepackage{xgreek}
\usepackage[Greek,Latin]{ucharclasses}
\setTransitionsForGreek{\setlanguage{greek}}{\setlanguage{english}}

%%-----------------------------------------------------------------------------
% Fonts
%%-----------------------------------------------------------------------------
\usepackage{fontspec}
\setmainfont{GFSDidot-Regular.ttf} % substitute with any font that exists on your system
\setsansfont{GFSDidot-Regular.ttf} % substitute with any font that exists on your system
\setmonofont{GFSDidot-Regular.ttf} % substitute with any font that exists on your system


%%-----------------------------------------------------------------------------
% Beamer Stuff
%%-----------------------------------------------------------------------------
\title{Ο χουλιγκανισμός στα σχολεία}
\author[Παιδαγωγικά]{XXXXXXX XXXXXXXX - 0000 \\ XXXXXXXX
XXXXXXXX  - 0000 \\ XXXXXXXX XXXXXXXX - 0000 \\ XXXXXXXX XXXXXXXX – 0000
\vspace{5.5pt} \\ 7ο εξάμηνο \\ Διδάσκων καθηγητής: XX XX
\\Νοέμβριος 2014 }
\date{}
%\vspace{-10.5pt}
% {\footnotesize \normalsize Εργασία στο
%μάθημα των Παιδαγωγικών} %

%\subtitle{MySQL}


%\setbeamersize{text margin left=1pt,text margin right=5pt} Universal left
% margin removal 
\let\olditem=\item%
\renewcommand{\item}{\olditem \justifying}%

\begin{document}

\begin{frame}[t]
% \newgeometry{top=0cm, left=0cm, right=0cm, bottom=0cm}
\vspace{1.5pt}
\centering\includegraphics[scale=0.5]{logo}
\vspace{5.5pt}
\titlepage
%\restoregeometry 
\end{frame} 

%\section{Ορισμός και ετυμολογία του όρου}

%
% First Slide
%

\begin{frame}[fragile,t]% for aligning text on the top of the slide
    \begin{minipage}[t]{0.99\textwidth}
      %  \vspace{0pt}
\frametitle{Ορισμός και ετυμολογία του όρου}
\vspace{-18.5pt} % reduce frame title after spacing
%% \makebox[\linewidth]{\rule{12cm}{3cm}} % a nice black box

%% Spacing between bullets and text
% \setlength\labelsep   {\dimexpr\labelsep + 0.5em\relax}  
% \setlength\leftmargini{\dimexpr\leftmargini + 0.5em\relax}
%%
\begin{adjustwidth}{-0.7cm}{} % reduce left spacing
\justifying
\begin{itemize} %\setlength{\itemsep}{15pt}
\item Ρατσιστική, ανάρμοστη, βίαιαη και
αντικοινωνική συμπεριφορά.\setlength{\itemsep}{15pt}
\item Συμμορίες του δρόμου - Λονδίνο δεκαετία
1980 - Patrick Hooligan. \setlength{\itemsep}{15pt}
\item Πολυπληθείς χειραγωγούμενες ομάδες ανθρώπων (Γκουστάβ Λε Μπόν) - ψυχολογία
της μάζας (Φρόυντ). \setlength{\itemsep}{15pt}
\item\setlength{\itemsep}{15pt}Εξάπλωση του φαινομένου σε Χώρες της Ευρώπης και
της Αμερικής.
\end{itemize}
\end{adjustwidth}
\end{minipage}
\end{frame}

%
% Second Slide
%

\begin{frame}[t]
\frametitle{Αίτια και μορφές του μαθητικού χουλιγκανισμού}
\vspace{-18.5pt}
\begin{adjustwidth}{-0.7cm}{}
\justifying
\begin{itemize}
\item Απότομη μετάβαση από την αυταρχική στην αντιαυταρχική εκπαίδευση των
τελευταίων τριανταπέντε ετών, χωρίς ενδιάμεσο στάδιο
προετοιμασίας.\setlength{\itemsep}{15pt}
\item Ανεπαρκή εκπαιδευτικά εγχειρίδια - λανθασμένα
εκπαιδευτικά συστήματα. \setlength{\itemsep}{15pt}
\item Επιβολή της γνώσης. \setlength{\itemsep}{15pt}
\item Εξάσκηση της απομνημονευτικής και όχι
της κριτικής ικανότητας των μαθητών. \setlength{\itemsep}{15pt}
\item Νεοφασισμός και κομματισμός στα σχολεία. \setlength{\itemsep}{15pt}
\end{itemize}
\end{adjustwidth}
\end{frame}

%
% Third Slide
%

\begin{frame}[t]
\frametitle{Αίτια και μορφές του μαθητικού χουλιγκανισμού}
\vspace{-18.5pt}
\begin{adjustwidth}{-0.7cm}{}
\justifying
\begin{itemize}
\item Αγόρια πιο επιθετικά από τα κορίτσια από προσχολική έως και εφηβική ηλικία
(15 έως 18 ετών).  \setlength{\itemsep}{15pt}
\item \textbf{Άμεση και συντελεστική επιθετικότητα} (ο μαθητής απειλεί τους
άλλους).
\setlength{\itemsep}{15pt}
\item \textbf{Άμεση και αντιδραστική  επιθετικότητα} (ο μαθητής απειλεί τους
άλλους για να πετύχει το στόχο του ή απειλεί όσους τον απειλούν). \setlength{\itemsep}{15pt}
\item \textbf{Έμμεση και συντελεστική επιθετικότητα} (ο μαθητής κουτσομπολεύει
και διαδίδει φήμες για τους άλλους). \setlength{\itemsep}{15pt}
\item \textbf{Έμμεση και αντιδραστική επιθετικότητα} (ο μαθητής όταν γίνεται
έξαλλος με τους άλλους, συχνά κουτσομπολεύει ή διαδίδει φήμες γι’ αυτούς).
\setlength{\itemsep}{15pt}
\end{itemize}
\end{adjustwidth}
\end{frame}

%
% Fourth Slide
%

\begin{frame}[t]
\frametitle{Αίτια και μορφές του μαθητικού χουλιγκανισμού}
\vspace{-18.5pt}
\begin{adjustwidth}{-0.7cm}{}
\justifying
\begin{itemize}
\item Εκπαίδευση από την οικογένεια ώστε να αντιδρούν επιθετικά.
\setlength{\itemsep}{15pt}
\item Eξώθησή τους λόγω κακοποίησης (ψυχικής ή σωματικής).
\setlength{\itemsep}{15pt}
\item Θυμός και απογοήτευση από πίεση για ανταπόκριση στα καθήκοντα της
καθημερινότητας. \setlength{\itemsep}{15pt}
\item Μίμηση αρνητικών προτύπων (ΜΜΕ).
\setlength{\itemsep}{15pt}
\item Οικονομική κρίση - ανέχεια, ανεπαρκής πρόνοια. \setlength{\itemsep}{15pt}
\item Θεοποίηση του χρήματος και των υλικών αγαθών. \setlength{\itemsep}{15pt}
\end{itemize}
\end{adjustwidth}
\end{frame}

%
% Fifth Slide
%

\begin{frame}[t]
\frametitle{Αίτια και μορφές του μαθητικού χουλιγκανισμού}
\vspace{-18.5pt}
\begin{adjustwidth}{-0.7cm}{}
\justifying
\begin{itemize}
\item Aποδοκιμάσμός και απορρίψη απο την σχολική κοινότητα ομάδων παιδιών που
παρουσιάζουν επιθετικές συμπεριφορές. \setlength{\itemsep}{15pt}
\item Σωματικοί, γνωστικοί και συναισθηματικοί προδιαθεσικοί παράγοντες.
\end{itemize}
\end{adjustwidth}
\end{frame}

%
% Sixth Slide
%

\begin{frame}[t]
\frametitle{Το φαινόμενο του εκφοβισμού (bullying)}
\vspace{-18.5pt}
\begin{adjustwidth}{-0.7cm}{}
\justifying
\begin{itemize}
\item Φαινόμενο επαναλαμβανόμενης, απρόκλητης, σκόπιμης και συνειδητής
  πρόθεσης πρόκλησης βλάβης ή φόβου σε κάποιον. \setlength{\itemsep}{15pt}
\item Έρευνα του Π.Ο.Υ. έδειξε οτι ένα ποσοστό από 9 \% ως και 73 \% παιδιών
επιδίδονται σε εκφοβισμό συμμαθητών τους. \setlength{\itemsep}{15pt}
\item Έρευνα  σε 60 Χώρες έδειξε οτι το 37,4 \% των παιδιών έχει υποστεί
εκφοβισμό τουλάχιστον μια φορά το μήνα !  \setlength{\itemsep}{15pt}
\item Ξεκινά με εκφοβισμό στην Δευτεροβάθμια εκπαίδευση, εξελίσσεται σε
παρενόχληση και βία στις διαπροσωπικές συναισθηματικές σχέσεις καθώς και σε ένα
πλήθος ακαδημαϊκών, κοινωνικών και ψυχοσωματικών συνεπειών.
\end{itemize}
\end{adjustwidth}
\end{frame}

%
% Seventh Slide
%

\begin{frame}[t]
\frametitle{Το φαινόμενο του εκφοβισμού (bullying)}
\vspace{-18.5pt}
\begin{adjustwidth}{-0.7cm}{}
\justifying
\begin{itemize}
\item Άμεσος εκφοβισμός \setlength{\itemsep}{15pt}
% nested bullets
\begin{itemize}
  \item Σωματικός εκφοβισμός
  \item Άμεσος λεκτικός εκφοβισμός
  \item Eκφοβισμός, που δεν είναι ούτε σωματικός ούτε λεκτικός.
\end{itemize}
\item Έμμεσος εκφοβισμός  \setlength{\itemsep}{15pt}
\item Διάφορες εγκληματικές συμπεριφορές. \setlength{\itemsep}{15pt}
\item 'Ισα ποσοστά κοριτσιών, εκφοβίζουν κορίτσια, καθώς και αγόρια που
εκφοβίζουν αγόρια ή κορίτσια. %\setlength{\itemsep}{15pt}
\end{itemize}
\end{adjustwidth}
\end{frame}

%
% Eight Slide
%

\begin{frame}[t]
\frametitle{Το φαινόμενο του εκφοβισμού (bullying)}
\vspace{-18.5pt}
\begin{adjustwidth}{-0.7cm}{}
\justifying
\begin{itemize}
\item Παριστάμενοι σε περιστατικά εκφοβισμού
\begin{itemize}
  \item Βοηθοί
  \item Ενισχυτές
  \item Παθητικοί θεατές
  \item Προασπιστές 
\end{itemize}
\end{itemize}
\end{adjustwidth}
\end{frame}

%
% Ninth Slide
%

\begin{frame}[t]
\frametitle{Αιτιολογικοί παράγοντες του σχολικού εκφοβισμού}
\vspace{-18.5pt}
\begin{adjustwidth}{-0.7cm}{}
\justifying
\begin{itemize}
\item Ατομικοί παράγοντες \setlength{\itemsep}{15pt}
\item Οικογενειακοί παράγοντες \setlength{\itemsep}{15pt}
\item Tο ίδιο το σχολείο
\end{itemize}
\end{adjustwidth}
\end{frame}

%
% Tenth Slide
%

\begin{frame}[t]
\frametitle{Ο κυβερνοεκφοβισμός}
\vspace{10pt}
\framesubtitle{Oρισμός του όρου}
\vspace{-18.5pt}
\begin{adjustwidth}{-0.7cm}{}
\justifying
\begin{list}{\quad}{}
\item Σύμφωνα με το Εθνικό Συμβούλιο Πρόληψης του Εγκλήματος των Η.Π.Α. - ο
κυβερνοεκφοβισμός - είναι η σκόπιμη και επαναλαμβανόμενη εχθρική συμπεριφορά,
που εκφραζόμενη μέσω απειλητικών μηνυμάτων και ψευδών εικόνων, δια του διαδικτύου ή των κινητών τηλεφώνων, αποσκοπεί στην πρόκληση ηθικής βλάβης ή στη δημιουργία προβλημάτων και φόβου από ένα άτομο ή ομάδα ατόμων, σε κάποιο άλλο άτομο, με
στόχο τη σπίλωση της υπόληψής του και την καταστροφή της κοινωνικής και επαγγελματικής
του ζωής.
\end{list}
\end{adjustwidth}
\end{frame}

%
% Eleventh Slide
%

\begin{frame}[t]
\frametitle{Ο κυβερνοεκφοβισμός}
\vspace{10pt}
\framesubtitle{Μορφές του κυβερνοεκφοβισμού}
\vspace{-18.5pt}
\begin{adjustwidth}{-0.7cm}{}
\justifying
\begin{list}{\quad}{}
\item \textbf{Μορφές κυβερνοεκφοβισμού}
\end{list}
\begin{itemize}
  \item Διαδικτυακή παρακολούθηση (Cyberstalking). \setlength{\itemsep}{15pt}
  \item Σπίλωση και δυσφήμιση (Denigration). \setlength{\itemsep}{15pt}
  \item Αποκλεισμός ή εξοστρακισμός από το διαδίκτυο (Exclusion).   \setlength{\itemsep}{15pt} 
  \item Φλόγισμα (Flaming), δηλαδή αποστολή προσβλητικού μηνύματος σε χυδαία  
  γλώσσα. \setlength{\itemsep}{15pt} 
  \item Βιντεοσκόπηση με το κινητό (Happy slapping). \setlength{\itemsep}{15pt}
  \item Διαδικτυακή παρενόχληση (Harassment). \setlength{\itemsep}{15pt}

\end{itemize}
\end{adjustwidth}
\end{frame}

%
% Twelfth Slide
%

\begin{frame}[t]
\frametitle{Ο κυβερνοεκφοβισμός}
\vspace{10pt}
\framesubtitle{Μορφές του κυβερνοεκφοβισμού}
\vspace{-18.5pt}
\begin{adjustwidth}{-0.7cm}{}
\justifying
\begin{itemize}
  \item Υποκλοπή και χρήση προσωπικού λογαριασμού (Impersonation ή Masquerade
  μεταμφίεση). \setlength{\itemsep}{15pt} 
  \item Δημοσιοποίηση προσωπικών στοιχείων (Outing).  \setlength{\itemsep}{15pt}
  \item Εμπαιγμός και εξαπάτηση (Trickery). \setlength{\itemsep}{15pt}
\end{itemize}
\begin{list}{\quad}{}
\item \textbf{Συνηθέστεροι τρόποι ηλεκτρονικής επικοινωνίας, για τη διάπραξη του
Kυβερνοεκφοβισμού}
\end{list}
\begin{itemize}
  \item Το ηλεκτρονικό ταχυδρομείο (e-mail).  \setlength{\itemsep}{15pt} 
  \item Tα άμεσα μηνύματα (Instant Messaging - IM)  \setlength{\itemsep}{15pt} 
\end{itemize}
\end{adjustwidth}
\end{frame}

%
% Thirteenth Slide
%

\begin{frame}[t]
\frametitle{Ο κυβερνοεκφοβισμός}
\vspace{10pt}
\framesubtitle{Μορφές του κυβερνοεκφοβισμού}
\vspace{-18.5pt}
\begin{adjustwidth}{-0.7cm}{}
\justifying
\begin{itemize}
  \item Tα μηνύματα γραπτού κειμένου σε κινητό (sms). \setlength{\itemsep}{15pt}
  \item Oι φωτογραφίες και τα βίντεο.  \setlength{\itemsep}{15pt}   
\end{itemize}
\begin{list}{\quad}{}
\item \textbf{Διαδικτυακοί τόποι, στους οποίους συνήθως λαμβάνουν χώρα περιστατικά
ψηφιακού εκφοβισμού}
\end{list}
\begin{itemize}
  \item Τα εικονικά δωμάτια συζήτησης (Chat rooms). \setlength{\itemsep}{15pt}
  \item Oι ιστότοποι κοινωνικής δικτύωσης (Social Networking Websites).  \setlength{\itemsep}{15pt} 
  \item Oι ηλεκτρονικοί πίνακες ανακοινώσεων (Message Boards).  \setlength{\itemsep}{15pt}  
 \end{itemize}
\end{adjustwidth}
\end{frame}

%
% Fourteenth Slide
%

\begin{frame}[t]
\frametitle{Ο κυβερνοεκφοβισμός}
\vspace{10pt}
\framesubtitle{Μορφές του κυβερνοεκφοβισμού}
\vspace{-18.5pt}
\begin{adjustwidth}{-0.7cm}{}
\justifying
\begin{itemize}
  \item Oι προσωπικοί ιστότοποι δημοσκοπήσεων ή ψηφοφοριών (Personal Polling Voting Websites).
 \setlength{\itemsep}{15pt}
\end{itemize}

\end{adjustwidth}
\end{frame}

%
% Fifteenth Slide
%

\begin{frame}[t]
\frametitle{Σύγκριση συμβατικού και ψηφιακού εκφοβισμού}
\vspace{-18.5pt}
\begin{adjustwidth}{-0.7cm}{}
\justifying
\begin{itemize}
  \item Ο ψηφιακός εκφοβισμός, δεν προϋποθέτει την υπεροχή που προϋποθέτει ο
  συμβατικός εκφοβισμός (δύναμη και ισχύ) για την επιβολή στο θύμα.
  \setlength{\itemsep}{15pt}
  \item Ο ψηφιακός εκφοβισμός, συνεχίζεται οπουδήποτε και οποτεδήποτε αρκεί το
  θύμα να να χρησιμοποιεί τις νέες τεχνολογίες, ενώ στον συμβατικό, σταματά όταν φύγει από
  το χώρο του σχολείου. \setlength{\itemsep}{15pt}
 \setlength{\itemsep}{15pt}
\end{itemize}

\end{adjustwidth}
\end{frame}

%
% Sixteenth Slide
%

\begin{frame}[t]
\frametitle{Αποτελεσματικές πρακτικές αντιμετώπισης του σχολικού
χουλιγκανισμού}
\vspace{-18.5pt}
\begin{adjustwidth}{-0.7cm}{}
\justifying
\begin{itemize}
  \item Σχεδιασμός από το σύνολο της σχολικής κοινότητας (δράσεις και στόχοι). 
\setlength{\itemsep}{15pt} 
 \item Διαμόρφωση πολιτικής. \setlength{\itemsep}{15pt}
 \item Εφαρμογή των στρατηγικών πρόληψης και παρέμβασης.
 \setlength{\itemsep}{15pt} 
 \item Διατήρηση του αποφασισθέντος προγράμματος  καταπολέμησης της ενδοσχολικής
 βίας και των βανδαλισμών. \setlength{\itemsep}{15pt} 
 \item Στρατηγικές πρόληψης \setlength{\itemsep}{15pt} 
\begin{itemize}
  \item H αναστοχαστική πρακτική (ανάλυση και αξιοποίηση όλων των δράσεων της
  σύγχρονης παιδαγωγικής, για την αποτροπή πράξεων της μαθητικής βίας).

\end{itemize}

\end{itemize}

\end{adjustwidth}
\end{frame}

%
% Seventeenth Slide
%

\begin{frame}[t]
\frametitle{Αποτελεσματικές πρακτικές αντιμετώπισης του σχολικού
χουλιγκανισμού}
\vspace{-18.5pt}
\begin{adjustwidth}{-0.7cm}{}
\justifying
\begin{itemize}
  \item
    \begin{itemize}
    \item Οι διαδραστικές πρακτικές.
    \end{itemize}
\end{itemize}    
\begin{list}{\quad}{}
\item \textbf{Δημιουργία κλίματος,που προάγει τις θετικές κοινωνικές σχέσεις
μεταξύ των μαθητών
}
\end{list}
\begin{itemize}
  \item Στρατηγικές πρόληψης \setlength{\itemsep}{15pt}  
  \begin{itemize}
  \item Η διάθεση τακτικού χρόνου για τη διαχείριση του προβλήματος (Circle
  time).  
  \item Η συνεργατική μάθηση. 
  \item Η χρήση της κοινωνιομετρίας (για την χαρτογράφηση των σχέσεων των παιδιών στην τάξη). 
  \item Το παιχνίδι ρόλων.  
  \item Οι  πρακτικές συμπαράστασης των θυμάτων από τους συμμαθητές τους. 
  \end{itemize}
  \item Στρατηγικές παρέμβασης \setlength{\itemsep}{15pt} 
    \begin{itemize}
  \item Στα παιδιά που εκφοβίζονται
    \end{itemize}
\end{itemize}



\end{adjustwidth}
\end{frame}



%
% Eighteenth Slide
%

\begin{frame}[t]
\frametitle{Αποτελεσματικές πρακτικές αντιμετώπισης του σχολικού
χουλιγκανισμού}
\vspace{-18.5pt}
\begin{adjustwidth}{-0.7cm}{}
\justifying
\begin{itemize}
  \item 
    \begin{itemize}
   \item Στα παιδιά που εκφοβίζουν. 
  \item Στους συνομηλίκους θεατές
  \item Στους εκπαιδευτικούς.  
  \item Στους γονείς.
  \item Στο περιβάλλον.
    \end{itemize}
\end{itemize}



\end{adjustwidth}
\end{frame}

\end{document}
